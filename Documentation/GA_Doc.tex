\documentclass{article}\usepackage[]{graphicx}\usepackage[]{color}
%% maxwidth is the original width if it is less than linewidth
%% otherwise use linewidth (to make sure the graphics do not exceed the margin)
\makeatletter
\def\maxwidth{ %
  \ifdim\Gin@nat@width>\linewidth
    \linewidth
  \else
    \Gin@nat@width
  \fi
}
\makeatother

\definecolor{fgcolor}{rgb}{0.345, 0.345, 0.345}
\newcommand{\hlnum}[1]{\textcolor[rgb]{0.686,0.059,0.569}{#1}}%
\newcommand{\hlstr}[1]{\textcolor[rgb]{0.192,0.494,0.8}{#1}}%
\newcommand{\hlcom}[1]{\textcolor[rgb]{0.678,0.584,0.686}{\textit{#1}}}%
\newcommand{\hlopt}[1]{\textcolor[rgb]{0,0,0}{#1}}%
\newcommand{\hlstd}[1]{\textcolor[rgb]{0.345,0.345,0.345}{#1}}%
\newcommand{\hlkwa}[1]{\textcolor[rgb]{0.161,0.373,0.58}{\textbf{#1}}}%
\newcommand{\hlkwb}[1]{\textcolor[rgb]{0.69,0.353,0.396}{#1}}%
\newcommand{\hlkwc}[1]{\textcolor[rgb]{0.333,0.667,0.333}{#1}}%
\newcommand{\hlkwd}[1]{\textcolor[rgb]{0.737,0.353,0.396}{\textbf{#1}}}%
\let\hlipl\hlkwb

\usepackage{framed}
\makeatletter
\newenvironment{kframe}{%
 \def\at@end@of@kframe{}%
 \ifinner\ifhmode%
  \def\at@end@of@kframe{\end{minipage}}%
  \begin{minipage}{\columnwidth}%
 \fi\fi%
 \def\FrameCommand##1{\hskip\@totalleftmargin \hskip-\fboxsep
 \colorbox{shadecolor}{##1}\hskip-\fboxsep
     % There is no \\@totalrightmargin, so:
     \hskip-\linewidth \hskip-\@totalleftmargin \hskip\columnwidth}%
 \MakeFramed {\advance\hsize-\width
   \@totalleftmargin\z@ \linewidth\hsize
   \@setminipage}}%
 {\par\unskip\endMakeFramed%
 \at@end@of@kframe}
\makeatother

\definecolor{shadecolor}{rgb}{.97, .97, .97}
\definecolor{messagecolor}{rgb}{0, 0, 0}
\definecolor{warningcolor}{rgb}{1, 0, 1}
\definecolor{errorcolor}{rgb}{1, 0, 0}
\newenvironment{knitrout}{}{} % an empty environment to be redefined in TeX

\usepackage{alltt}
\usepackage{amsmath,amssymb,amsfonts,amsthm}
\usepackage{eulervm}
\usepackage{upgreek}
\IfFileExists{upquote.sty}{\usepackage{upquote}}{}
\begin{document}
\title{GA Documentation}
\author{Mary Combs, Heejung Kim, Mohammad Soheilypour, Linqing(Waverly) Wei}
\maketitle

\section{Github}
User Name: WaverlyWei\\
Link: https://github.com/WaverlyWei/GA
\section{Contributions}
Group Memebers: Mary Combs, Heejung Kim, Mohammad Soheilypour, Linqing(Waverly) Wei\\

Mary: Implemented Selection function; Implemented Helper functions(GHFitness, calculateAIC); Wrote helper page for selection function\\

Heejung:Implemented initiation function; Wrote test cases\\

Mohammad:Implemented mutation function; Wrote helper information for initiation, crossover and mutation functions; Organized the strucutre of overall Select function\\

Waverly:Implemented crossover function; Wrote pdf documentation; Created Git Repo; Created initial R package
\section{Approach}
The solution is composed of three levels: Main function (Select), Modular functions(initiation, selection, crossover, mutation) and Helper functions(calculateAIC, GHFitness). \\
\section{Main function:Select}
Input: User defined dataset(data.frame), Model(lm/glm), Convergence Criterion, Maximum number of steps\\
Output: Selected variabels(list)\\
Stopping Criteria: AIC value converges \(\Delta AIC -> 0\)\\

Select function first calls Initiation function to get starting data. The starting data are separated into two components: parents and interncept. Since converge happens aymptotically in this solution, the limiting steps of converge is preset within Select function. Then it continues to the next step: breeding the next generation. To generate the next generation, according to Genetic Algorithm, four modular functions are involved. Parents are passed into Selection function, ranked based on objective function(fitness) and parent matrix is updtaed for the next steps. Children are then passed into Crossover and Mutation function to be randomized. The outcome is then set to be the next generation of parents and will go into another round.
\section{Modular Functions}
\subsection{Initiation}
Input: The number of independent variables and population size for each generation\\
Output: Initial matrix of parents\\

Initiation function creates a subset of original dataset as the starting generation for Select function. The starting matrix passed onto Select function contains both the initial matrxi and intercepts. \\
\subsection{Selection}
Input: Model matrix object with intercept and column for each independent variable specified in model, formula object ( eg. data$y ~ x1 + x2^2 + x2:x3 ), Parents matrix of P rows, Population size for each generation\\
Output: paretns selection, minimum AIC, parent with minimum AIC\\

Selection function assigns initital fitness probablity to each parent and then uses AIC as the objective function to rank parents. After sorting, fitness probablity is updated for each parent. Meanwhile, updated intercept and AIC value is also recorded together with selected parents.\\
\subsection{Crossover}
Input:P1, P2 (parent strings) and C as the number of variables\\
Output:P3,P4 (crossover strings, a list of two components)\\

Crossover function takes two parents strings and randomly choose a crossover site. Parents strings are cut at the site and ligated to produce children strings.
\subsection{Mutation}
Input:Parent, MutationProb, number of variables\\
Output: Mutated Parent \\

Mutation function takes a parent and mutates one or more sites according to the mutation probability.\\
\section{Helper Functions}
\subsection{GHFitness}
Assign fitness probablity to each parent based on the formula:
\begin{gather*}
\phi({v_i}^{(t)}) = \dfrac{2r_i}{P(P+1)}\\
\end{gather*}
This assignment gives the best inidvidual a probability of \(2/(P+1)\)
\subsection{calculateAIC}
AIC is calculated by applying linear model on response vector Y and covariates X. This function is used both for ranking individuals and for calculating the convergent AIC value (stopping criteria).

\section{Testing}
Testing is performed on four modular functions in two aspects: (1) Valid input (2)Expected return results
\subsection{Example 1: Simulated Data}
initData \(<-\) matrix(rnorm(2500, sd = 1:5), ncol = 5, byrow = TRUE)\\
initOutcome \(<-\) 1+-1*initData[,1]+2*initData[,3]+ 1.1*initData[,5]\\
expected output: c(1,2,4,6) \\
actual output: c(1,6)\\
(1: intercept, 2: X1, 3:X2, etc.)\\
\subsection{Example 2: Whitewine Quality Data}
Input: White Wine Quality Dataset\\
Output: c(1,2,3,4,5,7,8,9,11,12)\\
(Selected variables indexed by column number)\\
\begin{knitrout}
\definecolor{shadecolor}{rgb}{0.969, 0.969, 0.969}\color{fgcolor}\begin{kframe}
\begin{verbatim}
##   fixed.acidity volatile.acidity citric.acid residual.sugar chlorides
## 1           7.0             0.27        0.36           20.7     0.045
## 2           6.3             0.30        0.34            1.6     0.049
## 3           8.1             0.28        0.40            6.9     0.050
## 4           7.2             0.23        0.32            8.5     0.058
## 5           7.2             0.23        0.32            8.5     0.058
## 6           8.1             0.28        0.40            6.9     0.050
##   free.sulfur.dioxide total.sulfur.dioxide density   pH sulphates alcohol
## 1                  45                  170  1.0010 3.00      0.45     8.8
## 2                  14                  132  0.9940 3.30      0.49     9.5
## 3                  30                   97  0.9951 3.26      0.44    10.1
## 4                  47                  186  0.9956 3.19      0.40     9.9
## 5                  47                  186  0.9956 3.19      0.40     9.9
## 6                  30                   97  0.9951 3.26      0.44    10.1
##   quality
## 1       6
## 2       6
## 3       6
## 4       6
## 5       6
## 6       6
## [1] 4898   12
\end{verbatim}
\end{kframe}
\end{knitrout}
\end{document}
